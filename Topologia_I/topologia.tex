%%%%%%%%%%%%%%%%%%%%%%%%%%%%%%%%%%%%%%%%%%%%%%%%%%%%%%%%%%%%%%%%%%%%%%%%%
%
% Plantilla para libro de texto de matemáticas.
%
% Este documento es original de https://github.com/jmml97/LaTeX-Templates/blob/master/libro-matematicas/libro-%matematicas.tex pero se le han realizado algunas modificaciones
% Esta plantilla  ha sido desarrollada desde cero, utiliza algunas partes del código de la plantilla original %utilizada en apuntesDGIIM (https://github.com/libreim/apuntesDGIIM), basada a su vez en las
% plantillas 'Short Sectioned Assignment' de Frits Wenneker (http://www.howtotex.com),
% 'Plantilla de Trabajo' de Mario Román y 'Plantilla básica de Latex en Español'
% de Andrés Herrera Poyatos (https://github.com/andreshp). También recoge
% ideas de la plantilla 'Multi-Purpose Large Font Title Page' de
% Frits Wenneker y Vel (vel@latextemplates.com).
%
% Licencia:
% CC BY-NC-SA 4.0 (https://creativecommons.org/licenses/by-nc-sa/4.0/)
%
%%%%%%%%%%%%%%%%%%%%%%%%%%%%%%%%%%%%%%%%%%%%%%%%%%%%%%%%%%%%%%%%%%%%%%%%%

% ---------------------------------------------------------------------------
% CONFIGURACIÓN BÁSICA DEL DOCUMENTO
% ---------------------------------------------------------------------------

%\documentclass[11pt, a4paper, twoside]{article} % Usar para imprimir
\documentclass[10pt, a4paper]{article}

\linespread{1.3}            % Espaciado entre líneas.
\setlength\parindent{0pt}   % No indentar el texto por defecto.
\setlength\parskip{7pt}

% ---------------------------------------------------------------------------
% PAQUETES BÁSICOS
% ---------------------------------------------------------------------------

% IDIOMA
\usepackage[utf8]{inputenc}
\usepackage[spanish, es-tabla, es-lcroman, es-noquoting]{babel}

% MATEMÁTICAS
\usepackage{amsmath}    % Paquete básico de matemáticas
\usepackage{amsthm}     % Teoremas
\usepackage{amsfonts}
\usepackage{mathrsfs}   % Fuente para ciertas letras utilizadas en matemáticas

% FUENTES
\usepackage{newpxtext, newpxmath}   % Fuente similar a Palatino
\usepackage{FiraSans}                 % Fuente sans serif
\usepackage[T1]{fontenc}
\usepackage[italic]{mathastext}     % Utiliza la fuente del documento
                                    % en los entornos matemáticos

% MÁRGENES
\usepackage[margin=2.5cm, top=3cm]{geometry}


% LISTAS
\usepackage{enumitem}       % Mejores listas
\setlist{leftmargin=.5in}   % Especifica la indentación para las listas.

% Listas ordenadas con números romanos (i), (ii), etc.
\newenvironment{nlist}
{\begin{enumerate}
    \renewcommand\labelenumi{(\emph{\roman{enumi})}}}
  {\end{enumerate}}

%  OTROS
\usepackage{hyperref}   % Enlaces
\usepackage{graphicx}   % Permite incluir gráficos en el documento
\usepackage{multicol}   % Permite usar columnas

% ---------------------------------------------------------------------------
% COLORES
% ---------------------------------------------------------------------------

\usepackage{xcolor}     % Permite definir y utilizar colores

\definecolor{50}{HTML}{FCE4EC}
\definecolor{100}{HTML}{F8BBD0}
\definecolor{200}{HTML}{F48FB1}
\definecolor{300}{HTML}{F06292}
\definecolor{400}{HTML}{EC407A}
\definecolor{500}{HTML}{E91E63}
\definecolor{600}{HTML}{D81B60}
\definecolor{700}{HTML}{C2185B}
\definecolor{800}{HTML}{AD1457}
\definecolor{900}{HTML}{880E4F}

% ---------------------------------------------------------------------------
% DISEÑO DE PÁGINA
% ---------------------------------------------------------------------------

\usepackage{pagecolor}
\usepackage{afterpage}

% ---------------------------------------------------------------------------
% CABECERA Y PIE DE PÁGINA
% ---------------------------------------------------------------------------

\usepackage{fancyhdr}   % Paquete para cabeceras y pies de página

% Indica que las páginas usarán la configuración de fancyhdr
\pagestyle{fancy}
\fancyhf{}

% Representa la sección de la cabecera
\renewcommand{\sectionmark}[1]{%
\markboth{#1}{}}

% Representa la subsección de la cabecera
\renewcommand{\subsectionmark}[1]{%
\markright{#1}{}}

% Parte derecha de la cabecera
\fancyhead[LE,RO]{\sffamily\textsl{\rightmark} \hspace{1em}  \textcolor{500}{\rule[-0.4ex]{0.2ex}{1.2em}} \hspace{1em} \thepage}

% Parte izquierda de la cabecera
\fancyhead[RE,LO]{\sffamily{\leftmark}}

% Elimina la línea de la cabecera
\renewcommand{\headrulewidth}{0pt}

% Controla la altura de la cabecera para que no haya errores
\setlength{\headheight}{14pt}

% ---------------------------------------------------------------------------
% TÍTULOS DE PARTES Y SECCIONES
% ---------------------------------------------------------------------------

\usepackage{titlesec}

% Estilo de los títulos de las partes
\titleformat{\part}[hang]{\Huge\bfseries\sffamily}{\thepart\hspace{20pt}\textcolor{500}{|}\hspace{20pt}}{0pt}{\Huge\bfseries}
\titlespacing*{\part}{0cm}{-2em}{2em}[0pt]

% Reiniciamos el contador de secciones entre partes (opcional)
\makeatletter
\@addtoreset{section}{part}
\makeatother

% Estilo de los títulos de las secciones, subsecciones y subsubsecciones
\titleformat{\section}
  {\Large\bfseries\sffamily}{\thesection}{1em}{}

\titleformat{\subsection}
  {\Large\sffamily}{\thesubsection}{1em}{}[\vspace{.5em}]

\titleformat{\subsubsection}
  {\sffamily}{\thesubsubsection}{1em}{}

% ---------------------------------------------------------------------------
% ENTORNOS PERSONALIZADOS
% ---------------------------------------------------------------------------

\usepackage{mdframed}

%% DEFINICIONES DE LOS ESTILOS

% Nuevo estilo para definiciones
\newtheoremstyle{definition-style}  % Nombre del estilo
{}                                  % Espacio por encima
{}                                  % Espacio por debajo
{}                                  % Fuente del cuerpo
{}                                  % Identación
{\bf\sffamily}                      % Fuente para la cabecera
{.}                                 % Puntuación tras la cabecera
{.5em}                              % Espacio tras la cabecera
{\thmname{#1}\thmnumber{ #2}\thmnote{ (#3)}}  % Especificación de la cabecera

% Nuevo estilo para notas
\newtheoremstyle{remark-style}
{10pt}
{10pt}
{}
{}
{\itshape \sffamily}
{.}
{.5em}
{}

% Nuevo estilo para teoremas y proposiciones
\newtheoremstyle{theorem-style}
{}
{}
{}
{}
{\bfseries \sffamily}
{.}
{.5em}
{\thmname{#1}\thmnumber{ #2}\thmnote{ (#3)}}

% Nuevo estilo para ejemplos
\newtheoremstyle{example-style}
{10pt}
{10pt}
{}
{}
{\bf \sffamily}
{}
{.5em}
{\thmname{#1}\thmnumber{ #2.}\thmnote{ #3.}}

% Nuevo estilo para la demostración

\makeatletter
\renewenvironment{proof}[1][\proofname] {\par\pushQED{\qed}\normalfont\topsep6\p@\@plus6\p@\relax\trivlist\item[\hskip\labelsep\itshape\sffamily#1\@addpunct{.}]\ignorespaces}{\popQED\endtrivlist\@endpefalse}
\makeatother

%% ASIGNACIÓN DE LOS ESTILOS

% Teoremas, proposiciones y corolarios
\theoremstyle{theorem-style}
\newtheorem{nth}{Teorema}[section]
\newtheorem{nprop}{Proposición}[section]
\newtheorem{ncor}{Corolario}[section]
\newtheorem{lema}{Lema}[section]

% Definiciones
\theoremstyle{definition-style}
\newtheorem{ndef}{Definición}[section]

% Notas
\theoremstyle{remark-style}
\newtheorem*{nota}{Observación}

% AUXILIARE


% Ejemplos
\theoremstyle{example-style}
\newtheorem{ejemplo}{Ejemplo}[section]

% Ejercicios y solución
\theoremstyle{definition-style}
\newtheorem{ejer}{Ejercicio}[section]

\theoremstyle{remark-style}
\newtheorem*{sol}{Solución}

%% MARCOS DE LOS ESTILOS

% Configuración general de mdframe, los estilos de los teoremas, etc
\mdfsetup{
  skipabove=1em,
  skipbelow=1em,
  innertopmargin=1em,
  innerbottommargin=1em,
  splittopskip=2\topsep,
}

% Definimos los marcos de los estilos

\mdfdefinestyle{nth-frame}{
	linewidth=2pt, %
	linecolor= 500, %
	topline=false, %
	bottomline=false, %
	rightline=false,%
	leftmargin=0em, %
	innerleftmargin=1em, %
  innerrightmargin=1em,
	rightmargin=0em, %
}%

\mdfdefinestyle{nprop-frame}{
	linewidth=2pt, %
	linecolor= 300, %
	topline=false, %
	bottomline=false, %
	rightline=false,%
	leftmargin=0pt, %
	innerleftmargin=1em, %
	innerrightmargin=1em,
	rightmargin=0pt, %
}%

\mdfdefinestyle{ndef-frame}{
	linewidth=2pt, %
	linecolor= 500, %
	backgroundcolor= 50,
	topline=false, %
	bottomline=false, %
	rightline=false,%
	leftmargin=0pt, %
	innerleftmargin=1em, %
	innerrightmargin=1em,
	rightmargin=0pt, %
}%

\mdfdefinestyle{ejer-frame}{
	linewidth=2pt, %
	linecolor= 300, %
	backgroundcolor= 50,
	topline=false, %
	bottomline=false, %
	rightline=false,%
	leftmargin=0pt, %
	innerleftmargin=1em, %
	innerrightmargin=1em,
	rightmargin=0pt, %
}%

\mdfdefinestyle{ejemplo-frame}{
	linewidth=0pt, %
	linecolor= 300, %
	leftline=false, %
	rightline=false, %
	leftmargin=0pt, %
	innerleftmargin=1.3em, %
	innerrightmargin=1em,
	rightmargin=0pt, %
	innertopmargin=0em,%
	innerbottommargin=0em, %
	splittopskip=\topskip, %
}%

% Asignamos los marcos a los estilos
\surroundwithmdframed[style=nth-frame]{nth}
\surroundwithmdframed[style=nprop-frame]{nprop}
\surroundwithmdframed[style=nprop-frame]{ncor}
\surroundwithmdframed[style=ndef-frame]{ndef}
\surroundwithmdframed[style=ejer-frame]{ejer}
\surroundwithmdframed[style=ejemplo-frame]{ejemplo}
\surroundwithmdframed[style=ejemplo-frame]{sol}

% ---------------------------------------------------------------------------
% CONFIGURACIÓN DE LA PORTADA
% ---------------------------------------------------------------------------

\newcommand{\asignatura}{Topología I}

\newcommand{\autor}{Mapachana}

\newcommand{\grado}{2º Doble Grado en Ingeniería Informática y Matemáticas}

\newcommand{\universidad}{Universidad de Granada}

\newcommand{\enlaceweb}{github.com/Mapachana}

% ---------------------------------------------------------------------------
% CONFIGURACIÓN PERSONALIZADA
% ---------------------------------------------------------------------------

%%%%%%%%%%%%%%%%%%%%%%%%%%%%%%%%%%%%%%%%%%%%%%%%%%%%%%%%%%%%%%%%%%%%%%%%%%%%%
% ---------------------------------------------------------------------------
% COMIENZO DEL DOCUMENTO
% ---------------------------------------------------------------------------
%%%%%%%%%%%%%%%%%%%%%%%%%%%%%%%%%%%%%%%%%%%%%%%%%%%%%%%%%%%%%%%%%%%%%%%%%%%%%

\begin{document}

% ---------------------------------------------------------------------------
% PORTADA EXTERIOR
% ---------------------------------------------------------------------------

\newpagecolor{500}\afterpage{\restorepagecolor} % Color de la página
\begin{titlepage}

  % Título del documento
	\parbox[t]{\textwidth}{
			\raggedright % Texto alineado a la izquierda
			\fontsize{50pt}{50pt}\selectfont\sffamily\color{white}{
			  \textbf{\asignatura}
      }
	}

	\vfill

	%% Autor e información del documento
	\parbox[t]{\textwidth}{
		\raggedright % Texto alineado a la izquierda
		\sffamily\large\color{white}
		{\Large \autor }\\[4pt]
		\grado\\
		\universidad\\[4pt]
		\texttt{\enlaceweb}
	}

\end{titlepage}

% ---------------------------------------------------------------------------
% PÁGINA DE LICENCIA
% ---------------------------------------------------------------------------

\thispagestyle{empty}
\null
\vfill

%% Información sobre la licencia
\parbox[t]{\textwidth}{
  \includegraphics{by-nc-sa.pdf}\\[4pt]
  \raggedright % Texto alineado a la izquierda
  \sffamily\large
  {\Large Este libro se distribuye bajo una licencia CC BY-NC-SA 4.0.}\\[4pt]
  Eres libre de distribuir y adaptar el material siempre que reconozcas a los\\
  autores originales del documento, no lo utilices para fines comerciales\\
  y lo distribuyas bajo la misma licencia.\\[4pt]
  \texttt{creativecommons.org/licenses/by-nc-sa/4.0/}
}

% ---------------------------------------------------------------------------
% PORTADA INTERIOR
% ---------------------------------------------------------------------------

\begin{titlepage}

  % Título del documento
	\parbox[t]{\textwidth}{
			\raggedright % Texto alineado a la izquierda
			\fontsize{50pt}{50pt}\selectfont\sffamily\color{500}{
			  \textbf{\asignatura}
      }
	}

	\vfill

	%% Autor e información del documento
	\parbox[t]{\textwidth}{
		\raggedright % Texto alineado a la izquierda
		\sffamily\large
		{\Large \autor}\\[4pt]
		\grado\\
		\universidad\\[4pt]
		\texttt{\enlaceweb}
	}

\end{titlepage}

% ---------------------------------------------------------------------------
% ÍNDICE
% ---------------------------------------------------------------------------

\thispagestyle{empty}
\tableofcontents
\newpage

% ---------------------------------------------------------------------------
% CONTENIDO
% ---------------------------------------------------------------------------

\part{Teoría}

\section*{Fuentes}

La información para hacer estos apuntes se basa en los apuntes tomados en clase durante las lecciones de Topología I en el curso 2018/19 de 2º DGIIM.

\pagebreak

\section{Introducción}

\subsection{Nomenclatura}
Sea $X$ un conjunto y $x$ un elemento de $X$, entonces $x\in X$. Si $x$ no es un elemento de $X$, entonces se dice $x\not\in X$

Sea $A$ un subconjunto de $X$: $A\subset X$ (esta notación incluye contenido estricto o igual, puede ser $A=X$).

$\{x\}\subset X$ quiere decir que $\{x\}$ es un conjunto, en este caso unitario (un solo elemento) y está contenido en $X$.
$$x\in X \equiv \{x\}\subset X$$

\subsection{Operaciones con conjuntos}
\begin{ndef} \textbf{Operaciones de conjuntos.}
Sean $A,B$ subconjuntos de un conjunto $X$, entonces:
\begin{itemize}
\item Unión: $A\cup B=\{x\in X \diagup x\in A o x\in B\}$.
\item Intersección: $A\cap B=\{x\in X \diagup x\in A y x\in B\}$. Si no hay intersección se dice que es vacía, y se simboliza por $\emptyset$
\item Complementario: $X-A=\{x\in X\diagup x\not\in A$ siendo A subconjunto de X.
\end{itemize}
\end{ndef}

\begin{nprop}
\textbf{Propiedades.}
Sean $A,B,C\subset X$ conjunto:
\begin{itemize}
\item Asociativa:
\begin{equation}
\begin{split}
(A\cup B)\cup C = A\cup (B\cup C) \\
(A\cap B) \cap C = A\cap (B\cap C)
\end{split}
\end{equation}
\item Conmutativa:
\begin{equation}
\begin{split}
A\cup B = B\cup A \\
A\cap B = B\cap A
\end{split}
\end{equation}
\item Distributiva:
\begin{equation}
\begin{split}
A\cap(B\cup C) = (A\cap B)\cup (A\cap C) \\
A\cup (B\cap C) = (A\cup B)\cap (A\cup C)
\end{split}
\end{equation}
\item Propiedades del complementario:
\begin{itemize}
\item $A\cup (X-A) = X$
\item $A\cap (X-A) = \emptyset$
\end{itemize}
\end{itemize}
\end{nprop}

\begin{nth}
\textbf{Leyes de De Morgan}
De Morgan hizo dos afirmaciones, siendo $A,B\subset X$:
\begin{nlist}
\item $X-(A\cup B) = (X-A)\cap(X-B)$
\item $X-(A\cap B)= (X-A)\cup(X-B)$
\end{nlist}
\end{nth}
\begin{proof}Demostración de las Leyes de De Morgan.
\begin{nlist} 
\item Demostración de la primera propiedad:
\begin{equation}
\begin{split}
Sea\hspace{0.25cm}x\in X-(A\cup B) \Leftrightarrow x\not\in (A\cup B)\\
\Leftrightarrow x\not\in A \hspace{0.25cm}y\hspace{0.25cm} x\not\in B \\
\Leftrightarrow x\in X-A \hspace{0.25cm}y\hspace{0.25cm} x\in X-B \\
\Leftrightarrow x\in(X-A)\cap(X-B)
\end{split}
\end{equation}
\item Demostración de la segunda propiedad:
\begin{equation}
\begin{split}
Sea\hspace{0.25cm}x\in X-(A\cap B) \Leftrightarrow x\not\in (A\cap B)\\
\Leftrightarrow x\not\in A \hspace{0.25cm}o\hspace{0.25cm} x\not\in B \\
\Leftrightarrow x\in X-A \hspace{0.25cm}o\hspace{0.25cm} x\in X-B \\
\Leftrightarrow x\in(X-A)\cup(X-B)
\end{split}
\end{equation}
\end{nlist}
\end{proof}




\pagebreak

\section{Tema 1}

\begin{ejer}
Esto es un ejer.
Jeje.
\end{ejer}
\begin{sol}
Esto es la solucion.
jaja
\end{sol}
En muchas situaciones interesa hacer un estudio simúltaneo de más de un
carácter en una población. Así podemos hallar relaciones entre las distintas
características que estudiemos.





\end{document}
